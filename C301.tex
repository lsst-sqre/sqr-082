% This is the ADASS_template.tex LaTeX file, 19th Sep 2019.
% It is based on the ASP general author template file, but modified to reflect the specific
% requirements of the ADASS proceedings.
% Copyright 2014, Astronomical Society of the Pacific Conference Series
% Revision:  14 August 2014

% To compile, at the command line positioned at this folder, type:
% latex ADASS_template
% latex ADASS_template
% dvipdfm ADASS_template
% This will create a file called ADASS_template.pdf

\documentclass[11pt,twoside]{article}

% Do NOT use ANY packages other than asp2014. 
\usepackage{asp2014}

\aspSuppressVolSlug
\resetcounters

% References must all use BibTeX entries in a .bibfile.
% References must be cited in the text using \citet{} or \citep{}.
% Do not use \cite{}.
% See ManuscriptInstructions.pdf for more details
\bibliographystyle{asp2014}

% The ``markboth'' line sets up the running heads for the paper.
% 1 author: "Surname"
% 2 authors: "Surname1 and Surname2"
% 3 authors: "Surname1, Surname2, and Surname3"
% >3 authors: "Surname1 et al."
% Replace ``Short Title'' with the actual paper title, shortened if necessary.
% Use mixed case type for the shortened title
% Ensure shortened title does not cause an overfull hbox LaTeX error
% See ASPmanual2010.pdf 2.1.4  and ManuscriptInstructions.pdf for more details
\markboth{Sick}{The Documentation Developer Experience: Documentation Engineering at Rubin}

\begin{document}

\title{The Documentation Developer Experience: Documentation Engineering at Rubin}

% Note the position of the comma between the author name and the 
% affiliation number.
% Authors surnames should come after first names or initials, eg John Smith, or J. Smith.
% Author names should be separated by commas.
% The final author should be preceded by "and".
% Affiliations should not be repeated across multiple \affil commands. If several
% authors share an affiliation this should be in a single \affil which can then
% be referenced for several author names. If only one affiliation, no footnotes are needed.
% See ManuscriptInstructions.pdf and ASP's manual2010.pdf 3.1.4 for more details
\author{Jonathan Sick,$^1$}
\affil{$^1$J.Sick Codes Inc., Penetanguishene, Ontario, Canada; \email{jonathan@jsick.codes}}

% This section is for ADS Processing.  There must be one line per author. paperauthor has 9 arguments.
\paperauthor{Jonathan Sick}{jonathan@jsick.codes}{0000-0003-3001-676X}{J.Sick Codes Inc.}{}{Penetanguishene}{Ontario}{L9M 1Z6}{Canada}

% There should be one \aindex line (commented out) for each author. These are used to
% build up the author index for the Proceedings. The surname must come first, followed by
% initials. Note the use of ~ before each initial to control spacing.
% The \author entries (see above) have surname last. These \aindex entries have
% surname first.
% The Aindex.py command willl create them for you after you have constructed the \author
% The first entry should be the first author, for bold-facing the author index page-reference

%\aindex{FistAuthor1,~S.~A.}
%\aindex{Author2,~S.~B.}
%\aindex{Author3,~S.}


\begin{abstract}
At Rubin Data Management, we set out early in construction to create a healthy documentation culture.
In order to provide a good UX for documentation contributors, we developed a documentation platform that values low-friction documentation creation and guaranteed-accurate documentation techniques.
The basis is docs-like-code, with a versioned documentation hosting system that deploys documentation as a regular practice of merging, tagging, and making pull requests in a Git repository.
With automated templating, documentation projects are fast for anyone to start-up.
Documentation configuration and generation packages absorb complexity from authors.
A documentation portal updates automatically when content is pushed, and automatically curated on the basis of metadata published by each project.
This approach is used beyond code documentation, with a technical note system that enables developers to communicate systematically and out of the silos of individual teams.
% By using development tools such as GitHub Actions, Slack bots and Jupyter Notebooks, we are battling successfully the traditional view of documentation as the chore of last resort for both writers and readers.
\end{abstract}

% These lines show examples of subject index entries. At this stage these have to commented
% out, and need to be on separate lines. Eventually, they will be automatically uncommented
% and used to generate entries in the Subject Index at the end of the Proceedings volume.
% Don't leave these in! - replace them with ones relevant to your paper.
%\ssindex{FOOBAR!conference!ADASS 2019}
%\ssindex{FOOBAR!organisations!ASP}

% These lines show examples of ASCL index entries. At this stage these have to commented
% out, and need to be on separate lines. Eventually, they will be automatically uncommented
% and used to generate entries in the ASCL Index at the end of the Proceedings volume.
% The ascl.py command will scan your paper on possible code names.
% Don't leave these in! - replace them with ones relevant to your paper.
%\ooindex{FOOBAR, ascl:1101.010}

\section{Docs like code}

% - Enable the team to write Docs
% - Docs like code fits our team's workflow.
% - Reference Docs
% - Exploration of new technologies

To create a documentation platform that is appealing for our team to contribute to, we prioritized the developer experience.
\emph{Developer experience}, related to user experience, reflects how easy it is perform tasks and make contributions to a system (our system being the documentation set).
Although there are many valid frameworks for creating and publishing documentation, such as Wikis and PDF-centric document archives, at Rubin we were attracted to the \emph{docs-like-code} \citep{Gentle2022} framework.
It treats documentation as plain text files in version-controlled repositories (often the same repositories as related codebases) and published continuously to the web with same infrastructure that typically builds and tests software.
As a software development organization, particularly within the Data Management subsystem, a docs-like-code approach offers the best developer experience because it fits into our existing development workflows.
Code editors, GitHub pull requests, and Git branches and tags work equally well for code development as they do for docs development.

We have found benefits of treating docs-like-code beyond the streamlined workflow.
Since code repositories or other API-accessible systems contain the source-of-truth for many of the things we document, a docs-like-code framework facilitates building reference documentation directly from that source-of-truth.
This is commonly done for building Python API documentation, but at Rubin we have also build custom reference generators for our Kubernetes deployments (which are deployed from a GitHub repository, \url{https://phalanx.lsst.io}), pipelines, and services interfaces (using FastAPI where the data models for endpoints not only validate requests and responses, but also yield OpenAPI documentation).
A docs-like-code system also encourages the development of tooling (which is discussed in this work), and we have taken the opportunity to use the documentation platform as a low-risk testbed for new technologies.
Rubin's documentation platform has proven technologies like Kubernetes and Argo CD GitOps, and developed expertise and code libraries around Python libraries and services and Kafka event streaming that have been adopted throughout Rubin for mission-critical applications.

\section{Documentation deployment}

% Automatic deployment
% Features beyond read the docs

At Rubin, we created our own documentation deployment system, \textit{LSST the Docs} \citep{SQR-006}, to deploy and host our documentation.
This platform is designed so that documentation deployments happen automatically in the course of developing a Git repository, without any extra effort from the developer.
Each documentation projects is hosted at its own subdomain of \texttt{lsst.io}.
By default, when a project has changes merged to the default branch, new documentation content is pushed to this root URL.
When a branch or tag is pushed, corresponding documentation is published to a \texttt{/v/*/} subpath of that domain.
For branches, pull requests can reference this documentation as a preview.
For tags, this documentation becomes a permalink for that version of the docs.

\textit{LSST the Docs} is patterned after the existing Read the Docs service, but meets additional requirements.
Read the Docs provides its own build environment, but \textit{LSST the Docs} is compatible with any existing CI/CD system, including GitHub Actions and our self-hosted Jenkins platforms.
Documentation for the LSST Science Pipelines can only be be built in the bespoke Jenkins environment.

LSST the Docs is designed to be serverless from the end-users perspective, although an API service does coordinate documentation deployments from CI/CD systems.
All documentation assets are stored in an S3 object store, and the Fastly CDN provides routes user traffic to those S3 resources.

\section{Templates}

Initial project set up is an impediment to docs-like-code projects.
Not only do documentation projects require boilerplate infrastructure files that are outside the scope of contributors, they may also require configuration with outside systems like GitHub and web hosting that requires expert knowledge and permissions.
Templates mitigate this challenge so that any contributor can start a project without deep documentation platform expertise.

Rubin maintains templates for all types of software and documentation projects in a single repository.
The template system is built upon Cookiecutter, augmented with our own Templatekit package.
An example of each template is rendered into the repository, which provides a useful documentation reference for maintainers of projects.

By integrating templates with a Slack bot, templates are accessible to anyone in the Rubin organization.
Within Slack, a developer can configure values for a template, and the bot provisions the source repository, renders the template into it, and performs expert configuration on behalf of the developer (for example, registration with the documentation deployment system).
With this system, a team member is able to start a new code or documentation project and have it published to the web in a few minutes.

It is important to reduce the boilerplate code footprint in templates since any code can become technical debt.
We have had success in refactoring continuous integration and deployment workflows with GitHub Action's reusable workflows and custom actions.
Such works become centrally maintainable, and updates to deployment workflows often don't require any change in individual projects.

\section{Documentation hub}

The documentation deployment system enables us to have a large number of documentation sites.
The Rubin technical documentation hub provides a nexus for discovering content.
The documentation hub uses Algolia as its backend store.
Algolia provides full-text search, as well as filtering on metadata facets and sorting.
Additionally, Algolia provides a React Instantsearch library that substantially helped in creating the search front-end.

A key challenge in maintaining a documentation portal is ensuring that it remains up to date with the latest documentation content.
Rather than manually curating the portal, documentation updates are pushed to the portal on each update, and each documentation site also exports its own machine-readable metadata.
Our documentation crawling service, Ook, receives push events from updates on the documentation platform.
When it indexes content, it uses the metadata presented in that documentation to classify and sort the content.
The result is that documentation is curated from the bottom-up, and metadata curation is the realm of responsibility of individual teams for their documentation projects.

As of writing, the Algolia full-text search only includes documents hosted on lsst.io, but we intent to begin including multi-page guides and event documentation content from other platforms in the near future.

\section{Developer documentation}

Documentation is an essential ingredient of good developer experience.
Rubin Observatory Data Management maintains a developer guide to document and provide standards for all aspects of contributing to the Rubin Observatory codebases.
With respect to writing documentation, the Rubin Developer Guide provides technical guidance on the documentation tools and markup syntaxes.
In the Developer Guide, we also document \emph{how} to write documentation, such as the content style and content patterns.
% TODO cite baker if possible

\section{Project configuration}

Documentation tools like Sphinx can be overwhelming a casual documentation contributor to set up well.
Not only does Sphinx have a large number of configurations, but it also has an even more expansive ecosystem of extensions.
To alleviate this concern from individual projects, and also promote standardization across all our documentation projects, Rubin provides a pip-installable package for our Sphinx configuration, extension dependencies, custom extensions, and theming.
With a configuration module base, the conf.py files for individual projects is reduced to a single single.\footnote{This approach of providing an importable Sphinx configuration base is inspired by Astropy.}
The configuration system uses a TOML file to facilitate a simplified framework for per-project configurations.

% Whatever you abstract, you also own

\section{Technotes}

The documentation platform can be used for content beyond product or project documentation.
At Rubin Observatory, we've developed a notion of ``technotes'' as single-page websites that are deployed from their own GitHub repositories \citep{SQR-000}.
Technotes can be used for a wide range of purposes, from architectural design proposals, to research reports, and test reports.
Technotes solve the problem of making internal communications consistent and discoverable.
Rather than being in wiki, personal Google Docs, email, or Slack, technotes are indexed by the documentation portal.
Being websites, technotes are also available to the astronomy community, and we intend to make this content indexable with ADS.

We are now providing a Python package called \texttt{technote} that provides Sphinx theme extensions, and configurations that enable Sphinx generate these single-page articles.

\section{Conclusions}


\acknowledgements This material or work is supported in part by the National Science Foundation through Cooperative Agreement AST-1258333 and Cooperative Support Agreement AST1836783 managed by the Association of Universities for Research in Astronomy (AURA), and the Department of Energy under Contract No. DE-AC02-76SF00515 with the SLAC National Accelerator Laboratory managed by Stanford University.


\bibliography{C301}  % For BibTex

% if we have space left, we might add a conference photograph here. Leave commented for now.
% \bookpartphoto[width=1.0\textwidth]{foobar.eps}{FooBar Photo (Photo: Any Photographer)}

\end{document}
